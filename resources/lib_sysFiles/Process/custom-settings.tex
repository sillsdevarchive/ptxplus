% custom_tex.txt
%
% Custom TeX setup file for the ptx2pdf macro package.
% This file is loaded in after the auto_tex.txt settings
% file is compleated loading.




%%%%%%%%%%%%%%%%%%%%%%%  Extra Tweaks %%%%%%%%%%%%%%%%%%%%%%

% In a perfect world none of these would be needed but when
% your publication throws you a curve, perhaps one of these
% work-arounds might help you do what you want to do.


%%%%% Baselineskip Adjustment Hook
% This hook provides a means to adjust the baselineskip on a
% specific style. (Does not work with some markers like \fig)
% It provides a place to put the initial setting so the hook
% can make the change and then go back to the initial setting
% when done.
\newdimen\remblskip \remblskip=\baselineskip

% Baselineskip Adjustment Hook Example
%\sethook{start}{s1}{\remblskip=\baselineskip \baselineskip=10pt}
%\sethook{after}{s1}{\baselineskip=\remblskip}


%%%%% Skip a line in the working text
\def\skipline{\vskip\baselineskip}

%%%%% Skip just half a line in the working text (use carefully)
\def\skiphalfline{\vskip 0.5\baselineskip}

%%%%% Additional Marginal Verse Settings
% For marginal verse output we need to insert this right here.
% This should be temporary. It allows one stylesheet to undo
% superscript in a previous one It needs to go right here to
% keep everything else working
%
% This is needed to turn off superscript in the override stylesheet
%\catcode`\@=11
%\def\NoSuperscript  {\defp@ram{superscript}{false}}
%\catcode`\@=12
%
% In the override stylesheet (to follow) we need this to be
% inserted for \v:
% \Marker v
% \sethook{start}{v}{\kern-3sp}
% \FontSize 8
% \Bold
% \NoSuperscript

%%%%% Header output
% To adjust the size of the page number in the header or footer
% use the following code. Adjust font name and size as necessary.
%\font\mysmallfont="[../Fonts/CharisSIL/CharisSILB.ttf]" at 10pt
%\def\pagenumber{{\mysmallfont \folio}}

% This will output only the book name in the header that is
% found in \h. (This should be added to ptx2pdf.)
\catcode`\@=11
\def\bookname{\x@\extr@ctfirst\p@gefirstmark\relax\@book}
\catcode`\@=12

%%%%% Footnote tweaks
% Footnote caller kerning - To adjust space around the
% footnote caller use the following code Adjust the kern
% amounts as necessary
\let\OriginalGetCaller=\getcaller
\def\getcaller#1#2{%
  \kern0.2em\OriginalGetCaller{#1}{#2}\kern0.4em}

% Inter Note Skip - Adjust the horizontal space between footnotes,
% both paragraphed and non-paragraphed
\catcode`\@=11
  \intern@teskip=10pt
\catcode`\@=12

% Inter-note Penalty - Control the amount of "tension" between
% parts of a footnote to help control line breaking. If you
% use the highest setting, 10000, it will never break. A lower
% setting, like 9999, will lossen it up. Default is 9999.
\def\internotepenalty{9999}


%%%%% Quote Character Kerning
% Add some space between quote marks that are together.
%\quotekernamount = 0.1em

%%%%% Substituting Characters
% Some times, when a character does not exist in a font
% you can substitute from another if no special rendering
% is needed. This code will do that. Uncomment and modify
% as needed. This will put more cwi in your life :-)
%\font\cwi="[../Fonts/Padauk/Padauk.ttf]" at 10pt
%\catcode"A92E=\active				% Make U+A92E an active character
%\def^^^^a92e{\leavevmode{\cwi\char"A92E}}	% Define it to print itself  in \cwi

%%%%% Non-standard Spaces
% Some publications may use non-standard (U+0020) between words.
% But TeX (and XeTeX) will treat spaces other than U+0020 as
% non-breaking which messes up your justification. This is a
% work around to force TeX to break and stretch words with
% another space character in a controled way.
%\catcode"2009=13
%\def^^^^2009{\hskip .2em plus.1em minus.1em\relax}

%%%%% Heading space
% There always seems to be problems with extra space between
% the section heading and the top of the column when the
% section head is at the top of the column. To take up the
% slack this code will usually help. Any adjustments needed
% should be done in the .sty and \VerticalSpaceFactor.
% Trying to adjust this code doesn't seem to make any dif.
\catcode`\@=11
\def\gridb@x#1{%
 \setbox0=\ifgridp@c\vbox{\box#1}\else\killd@scenders#1\fi%
 \dimen2=\ht0 \advance\dimen2 by \dp0
 \dimen0=\baselineskip \vskip\baselineskip
 \ifgridp@c\line{}\nobreak\fi % otherwise first \line in loop won't get any baselineskip
							  % when doing a picture box, because it's not part of the
							  % current page
 \loop \ifdim\dimen0<\dimen2
   \advance\dimen0 by \baselineskip
	\line{}\nobreak \repeat
 \setbox0=\vbox to 0pt{\kern-\ht0\unvbox0}
 \unvbox0 \nobreak
}
\catcode`\@=12

%%%%% Get Rid of Over Full Warnings
% Surpress (if necessary) the vbox over full warning when
% when line spacing gets tight. If you use it edit the pt
% according to what you need. Keep in mind that you might
% be able to prevent an overful \vbox by changing your top
% and/or bottom margins.
%\vfuzz = 2.5pt

%%%%% Get Rid of Uggly Black Boxes
% This might help prevent the black boxes from appearing
% at the end of some lines if that is a problem. Edit as
% needed.
%\hfuzz=0.75pt

%%%%% Get more verbose output in the logs for debugging
%\tracingmacros=1
%\def\pton{\tracingparagraphs=1}
%\def\ptoff{\tracingparagraphs=0}


%%%%%%%%%%%%%%%%%%%%% TOC Generation %%%%%%%%%%%%%%%%%%%%%%

% The following code controls the layout of the TOC file
% This code can be addjusted to get the right look to your
% TOC. You can also adjust the \toc<n> styles found in
% the .sty file. The functions that are embedded in the
% TOC file are:
%	\tbltwowlheader{lable1}{label3}
%	\tbltwowlrow{val1}{val3}
% Or, if you wish a three column layout you could use this:
%	\tblthreewlheader{lable1}{label3}
%	\tblthreewlrow{val1}{val3}
% Other layouts, more code will be need to be added.
% To change the leader, leader spacing and other general
% layout formating, adjust the settings below.

\catcode`@=11

\def\makedigitsother{\m@kedigitsother}
\def\makedigitsletters{\m@kedigitsletters}

% These are custom settings that can go in the .tex file to
% adjust elements on the page
\def\myleader{.}		% The leader character
\def\leaderspace{0.6em}		% The space between the leader characters
\def\tblmarginright{0in}	% The width of the table's right margin
\def\tblheaderspace{4pt}	% The space between the header and the first row
\def\tblrowspace{4pt}		% The space between rows of the table
\def\tblthreewcolmngap{.75in}	% The space between col 1 & 2 on a three col table


% Macro code for leadered two colum layout:
\newdimen\tbltwowllabel
\def\tbltwowlheader#1#2{\parfillskip=\tblmarginright\toch #1\toch*\hfil
  \setbox0=\hbox{\toch #2\toch*}\tbltwowllabel=\wd0\box0\par\vskip \tblheaderspace}

\def\tbltwowlrow#1#2{\parskip=\tblrowspace\parfillskip=\tblmarginright\noindent\tocbn #1\tocbn*%
  \quad\leaders\hbox to \leaderspace{\hss\myleader\hss}\hfill\hbox to \tbltwowllabel{%
  \hfil\tocpg #2\tocpg*}\par}%

% Macro code for leadered three colum layout:
\newdimen\tblthreewllabel
\def\tblthreewlheader#1#2#3{\parfillskip=\tblmarginright\setbox0\hbox{\toch #1\toch*\relax}\tblthreewllabel=\wd0
  \noindent\box0\hskip\tblthreewcolmngap\toch #2\toch*\hfil\toch #3\toch*\par\vskip \tblheaderspace}

\def\tblthreewlrow#1#2#3{\parskip=\tblrowspace\parfillskip=\tblmarginright\noindent\hbox to \tblthreewllabel{\tocbn #1\tocbn*\hss\relax}%
  \hskip\tblthreewcolmngap\tocba #2\tocba*
  \leaders\hbox to \leaderspace{\hss\tocbn \myleader\tocbn*\hss}\hfil
  \enskip\tocpg #3\tocpg*\par}%

\catcode `@=12

%%%%%%%%%%%%%%%%%%%%%%%%%%%%%%%%%%% Map Handling %%%%%%%%%%%%%%%%%%%%%%%%%%%%%%%%%%

%%%%% Place The Map PDF File
% This will place a PDF map file on the typeset page
\catcode`@=11
\def\domap#1{\vbox to \the\textheight{\vfil\noindent\hfil\XeTeXpdffile #1 width \the\textwidth \hfil\par\vfil}\eject}%
\catcode `@=12


%%%%% Page Size

% This is controled from the project.conf file. However,
% as there is a wide range of paper sizes, we will store
% that information here (for now).

% Page/Paper Dimensions - Trimmed
% ISO A4 (default)
%	\PaperWidth=210mm	\PaperHeight=297mm	% 8.27 X 11.69 in.
% ISO B4
%	\PaperWidth=250mm	\PaperHeight=353mm	% 9.84 X 13.90 in.
% ISO A5
%	\PaperWidth=148mm	\PaperHeight=210mm	% 5.83 X 8.27 in.
% ISO B5
%	\PaperWidth=176mm	\PaperHeight=250mm	% 6.93 X 9.84 in.
% ISO A6
%	\PaperWidth=105mm	\PaperHeight=148mm	% 4.13 X 5.83 in.
% ISO B6
%	\PaperWidth=125mm	\PaperHeight=176mm	% 4.92 X 6.93 in.
% UBS Series 20
%	\PaperWidth=93mm	\PaperHeight=127mm
% UBS Series 30
%	\PaperWidth=100mm	\PaperHeight=145mm
% UBS Series 40
%	\PaperWidth=115mm	\PaperHeight=165mm
% UBS Series 50
%	\PaperWidth=120mm	\PaperHeight=180mm
% UBS Series 60 (5.71" x 8.27")
%	\PaperWidth=145mm	\PaperHeight=210mm
% UBS Series 70
%	\PaperWidth=162mm	\PaperHeight=220mm
% UBS Series 80
%	\PaperWidth=172mm	\PaperHeight=235mm
% UBS Series 90
%	\PaperWidth=210mm	\PaperHeight=295mm
% GPS typical size (5.25" x 8.25")
%	\PaperWidth=133mm	\PaperHeight=210mm
% GPS large size (5.8" x 8.7")
%	\PaperWidth=147mm	\PaperHeight=221mm
